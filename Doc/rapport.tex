\documentclass[french]{article}
\usepackage[T1]{fontenc}
\usepackage[utf8]{inputenc}
\usepackage{lipsum}
\usepackage{lmodern}
\usepackage{geometry}
\usepackage{babel}
\usepackage{graphicx}
\usepackage{lastpage}
\usepackage{ragged2e}
\usepackage[normalem]{ulem}
\usepackage{float}
\usepackage{import}
\usepackage{listings}

\setlength{\parindent}{0cm}

\geometry{
 	a4paper,
 	total={210mm,297mm},
 	left=20mm,
 	right=20mm,
 	top=25mm,
 	bottom=20mm,
}

\lstset{
    frame=single,
    breaklines=true,
    numbers=left,
    numbersep=6pt,
    tabsize=2,
    basicstyle=\small\sffamily
}

\usepackage{fancyhdr}
\pagestyle{fancy}

\lhead{Cotza Andrea \& Verdon Arthur}
\chead{}
\rhead{02.06.2016}
\cfoot{\thepage/\pageref{LastPage}}
\renewcommand{\headrulewidth}{0.4pt}
\renewcommand{\footrulewidth}{0.4pt}

\begin{document}
	\title{SER: Rapport laboratoire 3}
	\author{Cotza Andrea \& Verdon Arthur\\Prof. E. Lefrançois}
	\maketitle
    \vspace{5cm}
    \tableofcontents
    \newpage

    \section{Introduction}
    L'objectif de ce laboratoire consite [...]

    \section{Modification éventuelle de la structure XML}
    Pas de modification particulière apportée à la structure XML

    \section{Structure du programme}
    \subsection{XSL}
    La fichier XSL est découpé en plusieur templates.
    \\\\le premier des templates permet l'affichage des différents films et peut
    être appliqué dès que l'on veut afficher un film (il est d'ailleur appliquer
    dans les templates suivant).
    \\\\Trois autre template permettent de trier les films
    selon l'ordre voulu (par date de projection, par nom ou par classement/évaluation).
    Ces trois autre template applique le premier tempalte qui permet d'afficher la
    liste des films.
    \\\\Le dernier template permet d'afficher les acteurs dans le film et est appliqué
    dans le premier template s'occupant des films.

    \subsection{CSS}
    Afin de gagner en rapidité de dévellopement, il a été choisi d'utiliser un
    framework CSS/Javascript pour le design de la page HTML (générer depuis le XSL et le XML).
    Ce framework s'appel Bootstrap et nous a permis d'avoir un design correct pour la page.
    Peu de code CSS ont donc été nécessaire c'est pourquoi celui-ci ne sera pas
    ajouté dans les annexes.

    \subsection{Javascript}
    Comme dit précédement un framework à été utilisé pendant ce labo donc peu de
    JS à été écrit (pour les même raisons il ne sera pas ajouté dans les annexes).
    \\ En complèment a Bootrstrap, nous avons utilisé JQuery (qui est nécessaire
    pour Bootstrap) et une petite librairie (qui utilise Boostrap) qui permet
    d'afficher des étoiles pour les critiques.

    \section{Resultats}
    

    \subsection{Conclusion}


    \newpage

    \section{Annexe}
    \subsection{Code projections.xsl}
    \lstinputlisting[language=XSLT]{../Source/projections.xsl}

\end{document}
